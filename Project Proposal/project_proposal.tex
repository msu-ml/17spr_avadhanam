

\documentclass{sig-alternate-05-2015}

% Include useful packages
\usepackage{graphicx}
\graphicspath{ {images/} }
\usepackage{float}


\begin{document}

% Copyright
\setcopyright{acmcopyright}


\title{Approaches to Incorporating Whole Genome Information in the context of Mining Genetic Data}

\author{
% You can go ahead and credit any number of authors here,
% e.g. one 'row of three' or two rows (consisting of one row of three
% and a second row of one, two or three).
%
% The command \alignauthor (no curly braces needed) should
% precede each author name, affiliation/snail-mail address and
% e-mail address. Additionally, tag each line of
% affiliation/address with \affaddr, and tag the
% e-mail address with \email.
%
% 1st. author
Siddharth Avadhanam\\
       \email{avadhana@msu.edu}
}
% There's nothing stopping you putting the seventh, eighth, etc.
% author on the opening page (as the 'third row') but we ask,
% for aesthetic reasons that you place these 'additional authors'
% in the \additional authors block, viz.
\date{\today}
% Just remember to make sure that the TOTAL number of authors
% is the number that will appear on the first page PLUS the
% number that will appear in the \additionalauthors section.

\maketitle



\section{Problem Description}
In this work I propose to evaluate classical whole genome regression approaches in the statistical genetics literature such as lasso and
ridge-regression against newer ensemble methods that are being applied to data from Genome Wide Association Studies.
I will follow this review paper by Scyzymczak et al. \cite{Szymczak_machine_2009} and implement some of the more succesful approaches in a wheat data set.
I will evaluate these procedures in terms of cross validation accuracy, statistical considerations and computational performance.

The high-dimensionality and complexity of data coming in from Genome Wide Association Studies
poses a whole range of statistical and computational challenges in finding genetic covariates of important phenotypes
( disease, yield , height etc). These data commonly contain thousands of individuals with information on
hundreds of thousands of markers. \cite{zhang_chapter_2012} An important issue which has been given considerable attention in the statistical genetics
literature is how to choose a subset of these markers so that the analysis of these data with standard commodity hardware and statistical methods
becomes more tractable.\cite{de_los_campos_whole-genome_2013} Methods which analyze a single-marker at a time are popular, but are severly limited in that they do not take into account
interactions between markers, and come with their own set of challenges related to multiple testing.\cite{moskvina_multiple-testing_2008}
It is thus becoming increasingly important to explore methods that incorporate information form the whole genome and possibly from outside the domain of traditional statistics.
In this work I will take a first step in evaluating the popular approaches towards whole genome regression, and contrast it with newer ensemble methods
and possibly ( if time permits ) deep learning approaches. I will evaluate these methods on a publicly available dataset for wheat, looking
at statistical considerations such as robustness, cross-validation accuracy of prediction, and computational feasibility.

\section{Related Work}

There is considerable literature in statistical learning approaches towards the analysis of GWAS data. Earlier work has
touched upon single marker regression methods and the whole host of approaches that can be applied to
deal with the multiple-testing issues. Penalized regression methods such as ridge regression,\cite{austin_penalized_2013}
variable selection methods such as the lasso  and dimensionality reducing approaches such as principal components \cite{price_principal_2006} have all been applied and studied extensively in the analysis of
genome-wide data. Bayesian methods and mixed models approaches within specific communities such as animal breeding have been
studied and applied for a long time, and are increasingly being used with genome-wide data. \cite{de_los_campos_whole-genome_2013}
The discovery of gene-gene interactions is an active area of research where machine learning tools are being extensively employed \cite{upstill-goddard_machine_2013}

\section{Project Milestones}
This section will detail milestones so that I can roughly organize my work
toward the completion of this project.
\begin{itemize}
\item Prepare a proposal and survey important literature
\item Dig deeper into the literature and search for implementational details
\item Clean, organize and explore the data. Include or remove clinical covariates as necessary.
\item Prepare the code, run and test it extensively on example data. Decide if simulations will be necessary.
\item Prepare a second draft with a detailed introduction and some results based on exploration.
\item Run the alogrithms on the wheat data set, and collect the results. Prepare summaries and visualizations.
\item Prepare Final Report
\end{itemize}

\bibliography{cse_847}
\bibliographystyle{unsrt}
\end{document}
